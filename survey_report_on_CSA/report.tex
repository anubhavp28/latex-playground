\documentclass[11pt,oneside,a4paper]{article}
\usepackage{hyperref}
\title{Survey Report on Carry Save Adder}
\author{Anubhav Dinesh Patel (218) \\ \href{mailto:anubhavp28@gmail.com}{anubhavp28@gmail.com} \\
\textbf{IIIT Kalyani, West Bengal, India}
}

\usepackage{enumitem,kantlipsum}
\usepackage{xlop}

\begin{document}
	\maketitle
	\begin{enumerate}[wide, labelwidth=!, labelindent=0pt]
		\item \textbf{Introduction} \newline
		A carry-save adder is a type of digital adder, used in computer microarchitecture to compute the sum of three or more n-bit numbers in binary. It differs from other digital adders in that it outputs two numbers of the same dimensions as the inputs, one which is a sequence of partial sum bits and another which is a sequence of carry bits.
		\newline \newline
		There are many cases where it is desired to add more than two numbers together. The straightforward way of adding
together $m$ numbers (all $n$ bits wide) is to add the first two, then add that sum to the next, and so on. This requires
a total of $m - 1$ additions, for a total gate delay of $O(m * lg(n))$
(assuming lookahead carry adders). Using carry-save adder, a gate delay of $O(m + lg (n + m))$ can be achieved.
		\newline \newline
		Carry save adder is consists of three or more n-bit binary
numbers. Carry save adder is similar as full adder. Here we
are computing sum of 3-bit binary numbers, so we take 3 full
adders at first stage. Carry save unit consists of 6 full adders,
each of which computes single sum and carry bit based only
on the corresponding bits of the two input numbers \cite{jayasimha_lakshmi_kannan_daka_2017}. 
\newline \newline
Let X and Y are two 3-bit numbers and produces partial sum and carry
as S and C as shown in the Table 1.
\newline \newline
$$ Si = Xi \oplus Yi $$
$$ Ci = Xi + Yi $$
\centerline{Table 1. Carry save Adder Computation.}
\newline
\newline \newline
\centerline{
\begin{tabular}{cccccccc}
&X: & &1 &0 &0 &1 &1 \\
&Y: & &1 &1 &0 &0 &1 \\
+ &Z: & &0 &1 &0 &1 &1 \\
\hline
  &S: & &0 &0 &0 &0 &1 \\
+ &C: & &1 &1 &0 &1 &1 \\
\hline
&Sum: &1 &1 &0 &1 &1 &1
\end{tabular}}
		\item \textbf{Past Development} \newline
		While carry-save adders are rarely used for addition on modern general-purpose CPUs, they are used to construct Wallace Trees. Wallace Trees are combinatorial logic circuits used to multiply binary integers. They are a fast, efficient method to implement multiplication . Since
these adders do not propagate carry values between bits, they can produce multiplication products faster than other multiplication hardware\cite{dokachev_carpinelli_2015}.
		\newline \newline
		Carry-save adders are also used to construct circuits for modular multiplication. Most number-theoretic cryptosystems, such as RSA cryptosystems, are constructed based on modular multiplications. The design and VLSI implementation of fast algorithms for modular multiplication is a key to high-speed encryption/decryption cryptosystems \cite{introduction_to_biometric}.
		
		\item \textbf{Current Status} \newline
		Several faster combinatorial circuits for binary addition has been proposed, such as carry look-ahead adder (CLA) and carry skip adder (CSKA), which makes carry-save adders obsolete for binary additions. Carry-save adders have found use within circuitory for other types of computation (such as multipliation). Research effort going forward will be about using carry-save adders to implement combinatorial circuits for more complex computation than binary addition.
		 
		\item \textbf{Conclusion} \newline
		Though carry-save adders were developed for binary additon of more than 2 numbers, it is now mainly used as a building block of implementations of complex algorithms.
	\end{enumerate}		
	
\bibliography{test} 
\bibliographystyle{ieeetr}

\end{document}